\chapter{Cíle dizertační práce}\label{chap:cile}
Cíle předkládané dizertační práce jsou:
\begin{enumerate}[\bfseries 1.]
\item \textbf{Návrh algoritmu pro detekci novosti v datech s využitím adaptivních systémů}
\par 
Navrhovaný algoritmus pro detekci novosti bude využívat adaptivní systémy a bude mít interpretovatelný výstup. Jeho použití by mělo být možné v kombinaci libovolným adaptivním systémem. Algoritmus bude implementován v jazyce Python, který je v současné době jedním z nejrozšířenějších programovacích jazyků.
\item \textbf{Otestování navrženého algoritmu na syntetických datech}
\par 
Navržený algoritmus bude otestován na syntetických datech, která budou simulovat různé typy novosti a budou obsahovat šum. Mezi testovanými daty budou i data časových řad obsahujících trend. Výsledky algoritmu budou porovnány se obdobnými soudobými metodami využívajícími adaptivní systémy, konkrétně s algoritmem Learning Entropy a Error and Learning Based Novelty Detection.
\item \textbf{Provedení případových studií na reálných datech z oblasti biomedicíny a chemie}
\par 
Pro otestování navrženého algoritmu je zásadní provedení případových studiích na reálných datech. 
\item \textbf{Vyhodnocení kvality navrženého algoritmu z pohledu úspěšnosti detekce novosti} \par
Pro vyhodnocení kvality navrženého algoritmu budou určeny dosažené přesnosti detekce v různých scénářích. Dále budou pro vybrané scénáře určeny ROC křivky a vyhodnocena plocha pod nimi. Vyhodnocení bude provedeno pro scénáře s různými poměry signál-šum. Výsledky budou porovnány s výsledky algoritmů Learning Entropy a Error and Learning Based Novelty Detection. 

\end{enumerate}