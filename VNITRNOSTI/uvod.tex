
\chapter{Úvod}
\pagenumbering{arabic}
\setcounter{page}{1}

Tato disertační práce je věnována problematice využití adaptivních systémů při analýze dat. Vzhledem k exponenciálnímu celosvětovému nárůstu objemu dat \cite{expo} a ke zvyšování jejich variability roste i potřeba tato data analyzovat, kategorizovat a vytěžovat. Analýzou dat rozumíme proces, kdy z nezpracovaných naměřených dat získáme nějakou interpretovatelnou informaci, s kterou pak lze dále pracovat. Jedna z možných důležitých interpretací nově získaných dat je, zda-li se nově získaná data nějakým zásadním způsobem odlišují od předchozích dat. Této problematice se věnuje obor detekce novosti, neboli anomálií, který spadá do oblasti vytěžování dat a strojového učení. Úspěšná detekce novosti pak může být využita k vícero účelům. Například k diagnostice sledovaného procesu, ke změně struktury nebo parametrů adaptivního modelu za účelem zlepšení predikce, z konkrétních aplikací pak k odhalení neoprávněného vniknutí do sítě nebo zneužití dat, v lékařství se detekce novosti používá k diagnostickým účelům, z průmyslových aplikací pak k detekci poruchy a monitoringu stavu strojů, senzorů, ve zpracování textových dat k detekci nových témat, originálnosti textů atd. Spektrum využití je velice široké.
\par
V oblasti detekce novosti byla v posledních desetiletích intenzivního vývoje navrhnuta celá řada algoritmů. Vzhledem k rostoucímu výpočetnímu výkonu a rozmanitosti analyzovaných dat rostla i potřeba nových algoritmů. Nové algoritmy typicky předčily ostatní v rámci jedné aplikace, respektive v rámci jednoho typu dat.  Doposud se však nepodařilo vytvořit algoritmus, který by ve všech (nebo alespoň ve významné části) oblastech použití předčil již publikované algoritmy. I proto vznikají v oblasti detekce novosti neustále nové přístupy, které navíc umožňují analyzovat nové typy dat.

\section{Cíle dizertační práce}\label{chap:cile}
Cíle předkládané dizertační práce jsou:
\begin{enumerate}[label=\textbf{\arabic*})]
\item \textbf{Návrh algoritmu pro detekci novosti v datech s využitím adaptivních systémů}
\par 
Navrhovaný algoritmus pro detekci novosti bude využívat adaptivní systémy a bude mít interpretovatelný výstup. Jeho použití by mělo být možné v kombinaci libovolným adaptivním systémem. Algoritmus bude implementován v jazyce Python, který je v současné době jedním z nejrozšířenějších programovacích jazyků.
\item \textbf{Otestování navrženého algoritmu na syntetických datech}
\par 
Navržený algoritmus bude otestován na syntetických datech, která budou simulovat různé typy novosti a budou obsahovat šum. Mezi testovanými daty budou i data časových řad obsahujících trend. Výsledky algoritmu budou porovnány se obdobnými soudobými metodami využívajícími adaptivní systémy, konkrétně s algoritmem Learning Entropy a Error and Learning Based Novelty Detection.
\item \textbf{Provedení případových studií na reálných datech z oblasti biomedicíny a chemie}
\par 
Pro otestování navrženého algoritmu je zásadní provedení případových studiích na reálných datech. 
\item \textbf{Vyhodnocení kvality navrženého algoritmu z pohledu úspěšnosti detekce novosti} \par
Pro vyhodnocení kvality navrženého algoritmu budou určeny dosažené přesnosti detekce v různých scénářích. Dále budou pro vybrané scénáře určeny ROC křivky a vyhodnocena plocha pod nimi. Vyhodnocení bude provedeno pro scénáře s různými poměry signál-šum. Výsledky budou porovnány s výsledky algoritmů Learning Entropy a Error and Learning Based Novelty Detection. 

\end{enumerate}

\section{Členění práce}
Předkládáná disertační práce je členěna do sedmi kapitol, přičemž kapitoly \ref{chap:af_reserse}, \ref{chap:nd_reserse}, \ref{chap:gpd} jsou rešeršního charakteru. Druhá kapitola obsahuje přehled adaptivních filtrů a metod, které byly v rámci práce použity. Třetí kapitola je věnována přehledu různých metod detekce novosti a obsahuje i oblasti jejich využití.  Čtvrtá kapitola je věnována zobecněnému Paretovu rozdělení, které bylo použito v navrženém algoritmu detekce novosti.
\par  
Těžiště práce je v kapitolách \ref{chap:ese} a \ref{chap:vysledky}, které jsou autorovým příspěvkem k řešení problematiky detekce novosti s využitím adaptivních systémů.
Pátá kapitola obsahuje popis nově navrženého algoritmu nazvaném Extreme Seeking Entropy a krátce jsou zmíněny i jeho omezení. V šesté kapitole jsou výsledky tohoto algoritmu v porovnání s dalšími vybranými metodami adaptivní detekce novosti. Je zde také ukázána přesnost jeho detekce v různých scénářích a experimentální vyhodnocení času potřebného k výpočtu odhadu parametrů zobecněného Paretova rozdělení. Sedmá kapitola je závěrečnou a shrnuje dosažené výsledky.